\chapter{DESIGN}

How fulfill requirements

- Operation set (OGC-SFS)

- Batch Mode Processing

- Scalability Provided by:

-- Dataset centric (no need for record level locking, etc.) A good cluster based GIS processing environment needs to be separate from a graphical user interface so that it does not incur additional overhead on the compute nodes and so that it can interact well with the generally batch-scheduled environment of a cluster. In addition, it must be able to distribute data between all the nodes used such that a single node does not become a limiting factor in the environment's scalability. A variety of data distribution models should also be available so that the data is distributed in a manner consistent with the required processing. Of course, the environment should be able to execute all the standard geospatial operations as defined by the Open Geospatial Consortium in their Simple Features\cite{ogc-sf} standard.

-- Data parallel for speed/scaleup, different distribution models possible?

\section{HadoopGIS}

Uses JTS

Map/Reduce

“Communication” only possible in Reduce

Multiple Map/Reduce phases

How to use more than one dataset

Limited dataset distribution models

\section{ClusterGIS}

Uses GEOS

Split using MPI-IO

Communicate using MPI, possibility for different distributions

How a basic program works

How to use more than one dataset

Datasets could be distributed in any way imaginable.
