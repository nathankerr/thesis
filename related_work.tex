\chapter{RELATED WORK}

This thesis applies parallel processing techniques to the field of
GIS processing. I will first look at the state of GIS processing, then
Parallel Processing in general, then parallel GIS processing.

\section{GIS Processing}

Geographic Information Systems (GIS)\cite{gis} have been in use since the
1960's with the Canada Geographic Information System (CGIS)\cite{cgis}
and then moving from the mainframes to current desktop applications like
ESRI's ArcGIS\cite{arcgis} product, which began development in the 1980's.

In general, there are two types of GIS data: raster and vector. Raster
data is a set of cells, such as pixels in a picture, that have one or
more attributes (e.g., temperature, humidity, elevation). Each attribute
covers the entire area of the pixel. The entire raster dataset is
spatially located and states how much area is covered by each cell.

Vector data is comprised of spatially referenced geometric objects such
as points, lines, and polygons. Each object represents something in the
real world, and has associated attributes.

Most GIS processing systems are able to handle both raster and vector
data sources.

\subsection{Desktop GIS}

Desktop GIS packages such as ArcGIS\cite{arcgis}, QuantumGIS\cite{qgis},
and GRASS GIS\cite{grass} are commonly used for GIS processing and
analysis. While these programs provide graphical interfaces to their GIS
capabilities, their capabilities are limited by the computers they run
on. Datasets can be too large for their memories and computations can
take too long to be practical.  

\subsection{Database GIS}

An alternative approach to using these desktop programs is to employ a
geospatial database like PostGIS\cite{postgis}, ArcSDE\cite{arcsde} or Oracle
Spatial\cite{oracle}. Geospatial databases allow centralized access to, and
processing of, geospatial data through query languages such as SQL. As data is
stored and managed by the database software, advanced database features such
as indexes can be utilized to speedup data access and processing.

PostGIS is utilized as the core component of the Urban Systems
Frame-work\cite{usf} (USF) designed by the Digital
Phoenix\cite{digitalphoenix} project group at Arizona State University.
Digital Phoenix tries to integrate 3D visualization technology with simulated
and gathered GIS data to better understand the impacts of urban planning
decisions.

\subsection{GIS Simulation and Analysis}

UrbanSim

The popular urban simulation package,
UrbanSim\cite{urbansim} faces these same constraints.


GeoDa

PySal

\subsection{GIS Libraries}

The Open Geospatial Consortium (OGC) defined a core set of geospatial
processing operations in their Simple Features\cite{ogc-sf}
standard. These core operations allow for most geospatial processing
needs. One of the main libraries that provides these operations and
related data types is the Java Topology Suite\cite{jts} (JTS). Though
written in Java, the JTS has been used as the basis for ports into other
languages. The Geometry Engine, Open Source (GEOS) library is a C++
port of the JTS that also provides a C interface. PostGIS is implemented
using the GEOS library. The NetTopologySuite\cite{nts} (NTS) is a port
of the JTS into .NET.

As quality libraries are available for a variety of languages, there
is no need to re-implement the functionality they provide. This thesis
makes direct use of the JTS and GEOS libraries.

\section{Parallel Processing}

\subsection{Serial and Parallel Shared-Memory Applications}

Computer programs are executed by processors. The simplest of programs
is made up of a series of operations that are executed in order by
the processor. As this type of application is comprised by a series
of operations it is known as a serial program. Serial programs are not
able to take advantage of more than one processor. To utilize more than
one processor at a time, a set of serial programs that can communicate
with each other is required. Each serial program in this set is referred
to as a thread. Thus multi-threaded programs can utilize more than one
processing core. The simplest method of communication between threads is
to share a common section of memory. Therefore a parallel shared-memory
application could utilize at most the number of processors able to be
connected to a single section of memory.

Both serial and parallel shared-memory applications are limited to
a single computer, where computer means a group of processors that
are connected to the same memory. Most modern computers provide this
capability.

\begin{figure}
	%TODO
	\label{shared-memory}
	\caption{Shared-Memory Machine Architecture}
\end{figure}

\subsection{Parallel Distributed-Memory Applications}

To overcome the limitations of a single computer, the more scalable
architecture of a parallel distributed-memory machine was created. The
basic unit of this architecture is a processing element (PE) comprised
of one or more processors couple with memory. In other words, a PE is
a shared-memory machine. The PEs are interconnected using some sort
of networking technology. This architecture scales as well as the
interconnect does. Common interconnect technologies in use today are
Gigabit-Ethernet and InfiniBand. Clusters are parallel distributed-memory
machines.

\begin{figure}
	%TODO
	\label{distributed-memory}
	\caption{Distributed-Memory Machine Architecture}
\end{figure}

For a program to run on a parallel distributed-memory machine, it must
be able to work with multiple thread of operation where communication
between the threads goes over the interconnect. For the purposes of
discussion, threads running on different PEs are called tasks.

The main task in designing a parallel program is figuring out how to
split up the work between tasks. One method is to split up the processing
between tasks. Each task would generally have the same data and perform
variations of an operation on the data; for instance, running multiple
scenarios. This methodology is called Task Parallel. An example of task
parallelism is simulating the effects of various weather patterns on
an urban environment. Each task would have a copy of the environment
and run its variety of weather on it. The capability of executing each
scenario is limited to the capabilities of a single processing element,
but many scenarios can be executed at the same time.

Data Parallel methods split the data up between tasks an perform the
same, or similar, operations on each piece of data. To calculate the
effects of a weather pattern on an urban environment, the environment
would first be divided between the tasks with each task responsible for
one part of the environment. Each task would then calculate the effects
of the weather on its section of the environment, communicating with the
other PEs as needed to share information related to edge conditions, etc.

\section{Parallel GIS Processing}

Attempts have been made to overcome the limitations of single machine
implementations. Of primary interest are those extending current methods to
use multiple computers.

\subsection{Parallel Databases}

One method of using multiple computers to perform the required processing is
the use of parallel databases\cite{hpdb}. Parallel databases should be able to
spread both data storage and processing across multiple computers
transparently from the view of the SQL query programmer. Parallel database
techniques have been used to build a scalable geospatial database
system\cite{paradise,cs-paradise}.

Data is spread between computers using round-robin, hash, or spatial
partitioning. Because the data is able to be distributed between multiple
computers, the processing is able to scale to larger datasets. When a query is
processed across the database, a thread is created for each fragment of the
data. Thus as the data grows larger, the processing capabilities of the system
also increase.

Databases excel at working with indexed data while allowing multiple users to
interact with the data in a concurrently safe manner through the use of atomic
transactions. The requirements placed upon database systems to handle these
situations slow down computations that don't utilize indexes or work on an
entire dataset at once. The processing operations this research examines do
not require these restrictions, and as such a more efficient system can be
created.

\subsection{Problems with Desktop Programs on Clusters}

Current desktop approaches to GIS processing such as ArcGIS are unable to make
use of the multi-machine processing environment that compute clusters provide.
While reworking these programs to utilize these extended resources is
possible, it is non-trivial. GRASS GIS was reworked\cite{pgrass} to use its
collaboration features to distribute sub-queries among computers. The method
described in this paper utilizes multiple instances of GRASS in a master-slave
configuration where all participants access a shared data repository or
filesystem. The geometries are portioned between the various nodes. Operations
are done on the subsets, and the results are merged to produce the final
result.

While the method used to extend GRASS GIS will in fact speed up GIS
processing, it has two flaws. First, GRASS is designed to be used in an
interactive mode rather than a batch or script driven approach. Second, the
entire set of data used must still fit on a single computer to move it in or
out of the environment.

\subsection{Parallel Databases}

Parallel databases such as TeraData\cite{teradata} and Oracle\cite{oracle} use data parallel methods. Paradise\cite{paradise, cs-paradise} spreads data 
between computers using round-robin, hash, or spatial partitioning. Because
the data is able to be distributed between multiple computers, the
processing is able to scale to larger datasets.

When a query is processed across the database, a task is created
for each fragment of the data. Thus as the data grows larger, the
processing capabilities of the system also increase. If a particular
processing operation requires relatively less computation for each
data record this is fine. However, after the amount of computation per
data record increases beyond a certain point, which is dependent on the
speed of the machine used, the processing operation can be sped up by
utilizing more processors. Major factors in determining how much data
should be processed on each machine, and therefore how the data should
be spread between machines, are memory, computation, and communication
overhead to move the data and computation to another machine. The ratio
of computation to memory and commutation requirements is often referred
to as grain size. Coarser grained processes have more computation per
data record, while finer grained processes have little computation for
each data record.

Databases excel at working with indexed data while allowing multiple
users to interact with the data in a concurrently safe manner through
the use of atomic transactions. The requirements placed upon database
systems to handle these situations slow down computations that don't
utilize indexes or work on an entire dataset at once. The processing
operations this research examines do not require these restrictions,
and as such a more efficient system can be created.

Many universities and research institutions already have significant
investment in compute clusters. These clusters are groups of computer
linked together with high speed network interconnects and high performance
parallel filesystems such as Lustre\cite{lustre}. By separating compute
and storage resources at the cost of a high speed network, compute
clusters are able to separate computation from data storage.

Parallel filesystems allow the data to be separated from
the computer where the processing will be executed by spreading files
across multiple network connected fileservers allowing access that
can be faster than utilizing a computer's local disk for storage
while also enabling processing to spread across the available compute
resources based entirely on the process' grain size.

\subsection{Parallel GRASS GIS}

GRASS GIS was
reworked\cite{pgrass} to use its collaboration features to distribute
sub-queries among computers. The method described in this paper utilizes
multiple instances of GRASS in a master-slave configuration where
all participants access a shared data repository or filesystem. The
geometries are portioned between the various nodes. Operations are done
on the subsets, and the results are merged to produce the final result.

While the method used to extend GRASS GIS will in fact speed up GIS
processing, it has two flaws. First, GRASS is designed to be used in an
interactive mode rather than a batch or script driven approach.  Second,
the entire set of data used must still fit on a single computer to move
it in or out of the environment.
