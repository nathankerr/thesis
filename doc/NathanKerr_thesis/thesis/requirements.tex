\chapter{REQUIREMENTS}

\textbf{batch mode processing}

\textbf{data distribution options}

\textbf{standard geospatial operations}

\textbf{scalable, high performance}

\textbf{dataset centric}

A good cluster based GIS processing environment needs to be separate from
a graphical user interface so that it does not incur additional overhead
on the compute nodes and so that it can interact well with the generally
batch-scheduled environment of a cluster. In addition, it must be able to
distribute data between all the nodes used such that a single node does
not become a limiting factor in the environment's scalability. A variety
of data distribution models should also be available so that the data
is distributed in a manner consistent with the required processing. Of
course, the environment should be able to execute all the standard
geospatial operations as defined by the Open Geospatial Consortium in
their Simple Features\cite{ogc-sf} standard.

\section{easy to use to form new operations}

Sample program:

initClusterGIS

loadData (distributed/replicated)

process (user code here)

saveData (distributed/replicated)

finalizeClusterGIS