% Body of Research
%TODO: Introduction
%TODO: Related Work
%TODO: Requirements
%TODO: Design
%TODO: Experimental Setup
%TODO: Results
%TODO: Conclusion

\chapter{Introduction}

GIS processing is important

\section{Current methods}

Desktop GIS

Database GIS

\section{Significance}

Limitations of single processor implementations

\section{Attempted solutions}

Parallel DB

Parallel GRASS

Problems with desktop programs on clusters

\section{Needed solution, cluster programming model}

batch

mpi

scalable architecture

spmd

Thesis statement goes here

How this document goes about supporting the thesis statement.

\chapter{Related Work}

\section{Desktop GIS}

ArcInfo, QuantumGIS, GRASS GIS

\section{Database GIS}

PostGIS, Oracle Spatial, Client-Server Paradise

\section{GIS libraries}

JTS, GEOS

\section{Parallel DB}

Paradise

\section{Parallel GRASS}

Other converted programs?

\section{MRGIS MapReduce GIS}

\chapter{Requirements}

batch mode processing

data distribution options

standard geospatial operations

scalable, high performance

dataset centric

\section{easy to use to form new operations}

Sample program:

initClusterGIS

loadData (distributed/replicated)

process (user code here)

saveData (distributed/replicated)

finalizeClusterGIS

\chapter{Design/Implementation}

How to fulfill requirements

Architecture of solution

Split dataset across tasks (MPI I/O)

Each task works on their own part

MPI to communicate if needed

\chapter{Experimental Setup/Design}

How to show how well requirements were filled

\section{Operation set}

These operations are representative of problem space.

Data access based: parallelization is based off data decomp, so operations are too.

OGC standard requirements - survey of methods requiring 1, 2, or more data points.

\section{Dataset Description}

Full datasets; 34k employers, 1.2m parcels

sub datasets, how to generate from full

listing of datsets used

\section{Hardware setup}

Saguaro

\section{implementations}

\subsection{PostGIS}

\subsection{Hadoop Prototype}

\subsection{ClusterGIS}

\chapter{Results}

\section{Performance Analysis}

vary procs, maintain data

vary data, maintain procs

vary data, vary procs

\section{Comparisons}

PostGIS as baseline

PostGIS to Hadoop

Hadoop to ClusterGIS

PostGIS to ClusterGIS

\chapter{Recommendations}

Synthesis

\section{Applicable Problem Spaces}
Analysis of problem types that are good/bad for this approach

\chapter{Conclusion}

\section{Future Work}

additional decomposition methods, combined with alternative MPI communicators

addition of preprocessing methods (indexing, etc)

chunking of replicated dataset
