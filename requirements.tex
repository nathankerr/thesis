\chapter{REQUIREMENTS}

The chapter on related work teaches us what is needed for a good parallel GIS processing engine. Some of these requirements are derived from what makes good cluster based software, such as batch mode processing and scalability. This chapter defines the requirements and describes how to determine if an implementation meets them or not.

\section{Standard Geospatial Operations}

Any GIS processing application must have at least a core set of GIS processing operations.

The Open Geospatial Consortium (OGC) defines a set of such operations in their Simple Features\cite{ogc-sf} standard.

What is OGC?

History of SFS (Part 1 and various part 2s)

What does the SFS require (data, access)?

JTS, GEOS are open implementations of SFS-SQL

Measured by: yes or no

\section{Batch Mode Processing}

Measured by: yes or no

\section{Scalable}

Goal: Decrease processing time while using resources effectively.

vary procs, same data

vary data, same procs

vary data, vary procs

Measured by: speedup, scaleup

\section{Ease of Use}

Urban scientists != computer programmers, ease of use

subjective measure, but needs to be discussed.

Sample program:

initClusterGIS

loadData (distributed/replicated)

process (user code here)

saveData (distributed/replicated)

finalizeClusterGIS
