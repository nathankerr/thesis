\chapter{REQUIREMENTS}

Chapter 2 related the current state of parallel GIS processing, from which
the limitations of current approaches can be seen. From the capabilities
and limitations of current approaches, the requirements of a good parallel
GIS processing engine can be defined. Some of these requirements are
derived from what makes good cluster based software, such as batch mode
processing and scalability.

The main requirements of a parallel GIS processing engine are that is
supports standard geospatial operations, can be executed in a batch
environment, makes effective use of the additional resources provided
by the cluster environment, and is not too hard to use.

\section{Standard Geospatial Operations}

Any GIS processing application must have at least a core set of GIS
processing operations.  Without the capability to perform the required
processing, such an engine would be useless. The Open Geospatial
Consortium\cite{ogc} (OGC) defines a set of such operations in their Simple
Features\cite{ogc-sf} standard.

The Open Geospatial Consortium, Inc. (OGC) is a non-profit, international,
voluntary consensus standards organization that is leading the development
of standards for geospatial and location based services\cite{ogc}. The
OGC maintains a variety of standards that support GIS processing such
as cataloging, KML, WMS, and others.

The most important standard for this thesis is the Simple Features
Standard\cite{ogc-sfs} (SFS). The SFS defines ways of storing and
processing data including defining the Well Known Text (WKT) and Well
Known Binary (KWB) representations of geospatial data. Along with
data formats, a set of operations for working with geospatial data
are specified.

The OGC SFS is just a standard, and so it needs to be implemented.
The main opensource library implementing the standard is the Java
Topology Suite\cite{jts} (JTS).  While the JTS is implemented in Java,
it has been ported to other languages. For example PostGIS uses the C/C++
GEOS library\cite{geos}.

Compliance to this requirement can be tested by checking an independent
implementation against the standard, or assumed thought the use of a
compliant implimentation such as the JTS or GEOS libraries, assuming
full access to the library functionality is allowed.

\section{Batch Mode Processing}

As many clusters only allow non-interactive, or batch, processing modes,
the parallel GIS processing engine must fit this criteria.

To be compliant with this requirement, an implementation must support
at least the standard geospatial operations while in batch mode.

\section{Scalable}

The goal of a scalable application is utilize the resources given
it effectively. There are two aspects to scalability: capacity and
speedup. Capacity scalability means that an implementation is able to
solve a bigger problem than would otherwise be possible using a single
machine. Speedup measures the impact of using more resources as compared
to not using them.

Speedup is measured by comparing the processing time on a specific number
of processors ($n$) to the processing time on a single processor:

\[
\mbox{speedup}(n) = t(n)/t(1)
\]

To see how scalable an application is, a speedup plot is generated which
shows ideal speedup for the application based on a serial run. Figure
\ref{example_speedup} shows a sample speedup graph with three speedup
lines plotted. The Ideal Speedup line shows what the ideal speedup would
look like on the graph enabling easy comparison of the other line(s). The
80\% efficiency line shows an application that has achieved linear
speedup. This application scales very well. The diminishing returns
line shows an application that increases speedup to a point around 256
processors, at which time adding more resources actually increases the
program runtime. This limitation could occur because of limited data
size or some other concern.

\begin{figure}
	\label{example_speedup}
	\caption{Example Speedup Graph}
	\includegraphics{speedup.eps}
\end{figure}

Graphs like this will be used throughout this thesis to evaluate scalability.

\section{Ease of Use}

The last requirement is the most ambiguous: ease of use. The main thing
to think of here is that most people doing GIS processing are interested
in the results of the processing and would like the method to obtain
those results to be as easy as possible. In addition, most GIS processors
are not computer scientists, and so if a programmable interface is the
point of interaction, it must be as easy to use as possible so that the
learning curve is not excessive.

Ease of use is a subjective measure. No user studies will be performed
here, but it is a point of discussion that is important in evaluating
a GIS processing engine implementation.
