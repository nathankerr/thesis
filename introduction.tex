\chapter{INTRODUCTION}

%GIS processing is important, but data grows and computation time is too long

Geographic information systems (GIS) were designed to describe aspects of the
world around us. GIS data is made up of geometries such as points or a
polygons at certain longitudes, latitudes, and altitudes along with related
descriptive information such as a land use type. GIS data can be used to
represent a large range of real-world objects such as road or power networks,
building locations, or natural features such as lakes and rivers. Utilizing
GIS data is one method toward processing, analyzing, and simulating real-world
systems. One consumer application of GIS are the GPS based car navigation
systems common today.

Larger scale GIS applications also exist in areas such as city planning. City
planners use GIS to study road networks, zoning issues, and to simulate
population growth. As cities and metropolises grow, the amount of data
required to represent these areas also increases. As the amount of data
increases, so does the processing power required to complete the geospatial
analysis, processing, and simulation.

Geospatial (GIS) simulation and analysis are important processes to
understanding and improving our environment, both urban and natural.

% Data growth (with examples) leads to long runtime or inablility to process,
% long runtime creates a think wait think loop for the urban scientist

% Memory - can't store the entire dataset in RAM, swapping to disk time
% expensive

% Processor speed - only can do so much at one time

\section{Parallel computation methods}

% Spread data between computers, increasing memory capabilities

% More computers = more processors = doing more at one time

\section{Challenges with Parallel GIS Processing}

% Overview, full definintions in requirements section

% Speedup

% Scalability

% Programability/Flexability

\section {Alternative Parallel Approaches to GIS Processing}

% Thesis: impliment and evaluate.

% Map/Reduce with Hadoop

% Traditional Cluster: MPI

% What has been done/what is lacking. What is needed to consistute a good
% approach. Design of the two approaches. How to evaluate approches.
% Evaluation. Conclusion.

\section{Current Methods}

%Desktop GIS

Desktop GIS packages such as ArcGIS\cite{arcgis}, QuantumGIS\cite{qgis}, and
GRASS GIS\cite{grass} are commonly used for GIS processing and analysis. While
these programs provide graphical interfaces to their GIS capabilities, their
capabilities are limited by the computers they run on. Datasets can be too
large for their memories and computations can take too long to be practical.
The popular urban simulation package, UrbanSim\cite{urbansim} faces these same
constraints.

%Database GIS

An alternative approach to using these desktop programs is to employ a
geospatial database like PostGIS\cite{postgis}, ArcSDE\cite{arcsde} or Oracle
Spatial\cite{oracle}. Geospatial databases allow centralized access to, and
processing of, geospatial data through query languages such as SQL. As data is
stored and managed by the database software, advanced database features such
as indexes can be utilized to speedup data access and processing.

PostGIS is utilized as the core component of the Urban Systems
Frame-work\cite{usf} (USF) designed by the Digital
Phoenix\cite{digitalphoenix} project group at Arizona State University.
Digital Phoenix tries to integrate 3D visualization technology with simulated
and gathered GIS data to better understand the impacts of urban planning
decisions.

\section{Limitations}

%Limitations of single processor implementations

GIS data has become easier to gather in recent years with the proliferation on
low cost GPS devices. The availability of these devices significantly reduces
the cost of data collection, moving it from a government funded service to the
capabilities of private companies and individuals even to the point of open
source style maps such as OpenStreetMap\cite{openstreetmap}.

With the reduced cost of data collection, the amount of data has grown
significantly. As datasets grow and the associated processing becomes more
sophisticated, the limitations of programs that only work on a single computer
are felt through long processing times and inability to work with the required
data. Thus a method of utilizing multiple computers to complete the required
processing is needed to increase the size of the data that can be processed
and reduce the time required to complete the processing.

\section{Attempted solutions}

Attempts have been made to overcome the limitations of single machine
implementations. Of primary interest are those extending current methods to
use multiple computers.

\subsection{Parallel Databases}

One method of using multiple computers to perform the required processing is
the use of parallel databases\cite{hpdb}. Parallel databases should be able to
spread both data storage and processing across multiple computers
transparently from the view of the SQL query programmer. Parallel database
techniques have been used to build a scalable geospatial database
system\cite{paradise,cs-paradise}.

Data is spread between computers using round-robin, hash, or spatial
partitioning. Because the data is able to be distributed between multiple
computers, the processing is able to scale to larger datasets. When a query is
processed across the database, a thread is created for each fragment of the
data. Thus as the data grows larger, the processing capabilities of the system
also increase.

Databases excel at working with indexed data while allowing multiple users to
interact with the data in a concurrently safe manner through the use of atomic
transactions. The requirements placed upon database systems to handle these
situations slow down computations that don't utilize indexes or work on an
entire dataset at once. The processing operations this research examines do
not require these restrictions, and as such a more efficient system can be
created.

\section{Problems with Desktop Programs on Clusters}

Current desktop approaches to GIS processing such as ArcGIS are unable to make
use of the multi-machine processing environment that compute clusters provide.
While reworking these programs to utilize these extended resources is
possible, it is non-trivial. GRASS GIS was reworked\cite{pgrass} to use its
collaboration features to distribute sub-queries among computers. The method
described in this paper utilizes multiple instances of GRASS in a master-slave
configuration where all participants access a shared data repository or
filesystem. The geometries are portioned between the various nodes. Operations
are done on the subsets, and the results are merged to produce the final
result.

While the method used to extend GRASS GIS will in fact speed up GIS
processing, it has two flaws. First, GRASS is designed to be used in an
interactive mode rather than a batch or script driven approach. Second, the
entire set of data used must still fit on a single computer to move it in or
out of the environment.

\section{Cluster Programming Model}

Many universities and research institutions already have significant
investment in compute clusters. These clusters are groups of computer linked
together with high speed network interconnects and high performance parallel
filesystems such as Lustre\cite{lustre}.

% batch

Cluster access is a valuable resource, and is treated as such. To maximize
usage, cluster resources are allocated through batch queuing systems. In
general, interactive access is not allowed. Therefore, programs that run on
clusters need to be able to work without user input at the time of execution.

% mpi

Programs that run on clusters require a method of utilizing the computers
allocated. A common method in use today is message passing using MPI. MPI is a
standard with various different implementations, one of which is installed on
most clusters. Moving from one MPI implementation to another usually just
requires recompiling the program.

% scalable architecture

Cluster based programs can be evaluated in terms of speedup and scalability.
Speedup is the amount of extra processing power gained by utilizing multiple
processors. Scalability describes the program's ability to operation on
various numbers of processors utilizing various amounts of data.

\section{Cluster based approaches to GIS}

% Thesis statement goes here

This thesis implements and evaluates two parallel, dataset centric approaches
to processing large geospatial datasets on clusters. The first approach,
called HadoopGIS, uses the Hadoop\cite{hadoop} map/reduce framework. The
second approach, ClusterGIS, uses the more traditional approach to programming
for clusters, MPI. Both approaches provide the geospatial operations required
by the Open Geospatial Consortium's Simple Features\cite{ogc-sf} standard. By
applying data parallel programming methods and dispensing with record centric
processing methods, both these methods create a fairly easy environment to
program in while providing significant speedup and scaleup.

% How does this document support

This document first discusses related works in GIS processing and
parallization. These related works are used to form the requirements of a good
cluster based implementation. These requirements are then transferred into the
design of HadoopGIS and ClusterGIS. After the implementation design discussion
is concluded, the evaluation method is discussed in the experimental setup.
From the results of the experiments, the performance of the implementations
are evaluated and analyzed in the results section. Then I provide some
recommendations and conclusions.